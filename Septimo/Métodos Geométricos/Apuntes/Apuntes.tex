\documentclass[a4,12pt]{aleph-notas}
% Se recomienda leer la documentación de esta
% clase en https://www.alephsub0.org/recursos/

% -- Paquetes adicionales
\usepackage{aleph-comandos}
\usepackage{enumitem}
\usepackage{amssymb}
\usepackage{ragged2e}
\usepackage{multicol}

% -- Datos de las notas
\universidad{Universidad Nacional de Colombia}
\autor{Angel Almonacid}
\materia{Mécánica Cuántica}
\nota{Apuntes}
\fecha{\today}

\longtitulo{0.65\linewidth}
\logouno[3.0cm]{Logos/LogoUnal}
\logodos[2.0cm]{Logos/IconoFisica}


% -- Comandos adicionales


\begin{document}

\encabezado

\section{Función de Onda}
\subsection{Ecuación de Schrodinger}
Se parte de una hipótesis que cumpla $p(\vec{x},t)d^3x=\mid \psi(\vec{x},t)\mid^2d^3x$, siendo $\psi(\vec{x},t)$ la función que describe a la onda estudiada. De esta forma vamos a buscar que la función de onda cumpla cuatro condiciones:\\
\begin{itemize}
    \item $\frac{d^2x}{dt^2}=F/m$ Ecuación de movimiento \item $\frac{\partial}{\partial t} \psi(\vec{x},t)=L\psi+c$ Ecuación diferencial lineal \item $\int \left\lvert \psi(\vec{x},t) \right\rvert^2 d^3x=1 $ Función normalizada \item $\psi( \vec{x},t)=Ae^{i(\vec{k}\cdot \vec{x}-w\cdot t)}$ Forma sinusoidal
\end{itemize}

Así, podemos formular una hipótesis que además de las condiciones previas vuelva adimensional el exponente de la función

\begin{equation*}
    \psi(\vec{x},t)=Ae^{\frac{i}{\hbar}\left( \vec{p} \cdot \vec{x}- \frac{p^2}{2m}t\right)}
\end{equation*}

Ingresaremos la hipótesis en las ecuaciones planteadas para terminar de formular la función de onda

\begin{align*}
    \frac{\partial \psi}{\partial t}=\frac{-ip^2}{2\hbar m} \psi && \nabla\psi=\frac{i}{\hbar}\vec{p}\psi \rightarrow \nabla^2\psi=-\frac{p^2}{\hbar^2}\psi
\end{align*}

\begin{equation}\label{Schrodinger}
    i\hbar\frac{\partial}{\partial t} \psi=-\frac{\hbar^2}{2m}\nabla^2\psi
\end{equation}

Así, obtenemos la ecuación de Schrödinger para una función de onda plana. Si sabemos que una onda de esta forma es solución de la ecuación diferencial, una combinación lineal de estas ondas planas también lo será, así una solución general puede describirse de la forma (de manera discreta y continua respectivamente)
    $$\psi_k(\vec{x},t)=\sum_{1}^{n}e^{\frac{i}{\hbar}\left(\vec{p}\cdot\vec{x}-\frac{p^2}{2m}t\right)}\varphi(p)\frac{d^3 p}{(2 \pi \hbar)^3}$$
\begin{equation}\label{Superposition}
    \psi(\vec{x},t)=\int e^{\frac{i}{\hbar}\left(\vec{p}\cdot\vec{x}-\frac{p^2}{2m}t\right)}\varphi(p)\frac{d^3 p}{(2 \pi \hbar)^3}
\end{equation}

Donde $\varphi(p)$ es una función de peso para cada momento $p$, esta función de peso nos da información sobre la distribución del paquete de ondas. Para una distribución gaussiana, podemos hacer el cálculo de la constante de normalización $A$ y de la función de onda $\psi(\vec{x},t)$.\\\\
Podemos definir la densidad de probabilidad $\rho(\vec{x},t)=\left\vert \psi(\vec{x},t) \right\vert^2 $ y el valor esperado $\langle x\rangle=\int_{-\infty}^{\infty}\left\vert \psi \right\vert^2 x dx$. Si deseamos hallar la desviación cuadrática media, sabemos que desde el estudio estadístico se define como $\left(\Delta x^2\right)=\sum_{i}^{n} \left(x-\bar{x}\right)^2$, así podemos definir la desviación cuadrática media como\\
 \begin{equation}
    \left( \Delta x^2\right)=\langle(x-\langle x\rangle)^2 \rangle
 \end{equation}
\section{Espacio de momentos}

Vamos a realizar la transformación al espacio de momentos a través de la transformada de Fourier, así podemos definir la siguiente transformación:

\begin{align*}
    \left\vert \psi (\vec{x},t)\right\vert^2 d^3x \longrightarrow \mathbf{W} (\vec{p},t)d^3p && \int_{V} \mathbf{W}(\vec{p},t)d^3p=1
\end{align*}

La relación de transformación a través de la transformada integral de Fourier se define de la siguiente forma
\begin{equation}\label{TFourier}
    \psi(\textbf{x},t)=\int\frac{d^3p}{(2\pi\hbar)^3}\varphi(\textbf{p},t)e^{\frac{i}{\hbar}(\textbf{p}\cdot\textbf{x})}
\end{equation}

Siendo $\varphi(\textbf{p},t)$ la transformada de Fourier de la función $\psi(\textbf{x},t)$. Para encontrar $\varphi$ debemos hallar la función inversa de Fourier, multiplicaremos por $e^{-\frac{i}{\hbar}p'x}$ e integraremos con respecto a x \\ \\ \\
\begin{align*}
    \int d^3x \psi(\textbf{x},t)e^{-\frac{i}{\hbar}\textbf{p}'-\textbf{x}}=\int d^3p\varphi(\textbf{p},t) \frac{1}{(2\pi\hbar)^3}\int d^3x e^{\frac{i}{\hbar}(\textbf{p}-\textbf{p}')\cdot \textbf{x}}=\int d^3p \varphi(\textbf{p},t) \delta(\textbf{p}-\textbf{p}')=\varphi(\textbf{p}',t)
\end{align*}
Obtenemos como resultado la siguiente relación de transformación inversa, identificando la delta de Dirac dentro del termino de la integral con respecto a p
\begin{equation}\label{IFourier}
    \varphi(\textbf{p},t)=\int d^3x\psi(\textbf{x},t)e^{-\frac{i}{\hbar}(\textbf{p}\cdot\textbf{x})}
\end{equation}

\subsection{Densidad de probabilidad en el espacio de momentos}
Usaremos el mismo método para calcular la función de onda en el espacio de momento para calcular la densidad de probabilidad en este mismo espacio
\begin{align*}
\left\vert \psi(\textbf{x})\right\vert^2&=\int \frac{d^3 p}{(2 \pi \hbar)^3}\int \frac{d^3 p'}{(2 \pi \hbar)^3} \varphi(\textbf{p},t)\varphi^{\ast}(\textbf{p}',t) e^{\frac{i}{\hbar}(\textbf{p}-\textbf{p}')\cdot \textbf{x}}\\
\int \left\vert \psi(\textbf{x})\right\vert^2 d^3 x &= \int \frac{d^3 p}{(2 \pi \hbar)^3}\int \frac{d^3 p'}{(2 \pi \hbar)^3} \varphi(\textbf{p},t)\varphi^{\ast}(\textbf{p}',t)\int e^{\frac{i}{\hbar}(\textbf{p}-\textbf{p}')\cdot x} d^3 x\\
&=\int \frac{d^3 p}{(2 \pi\hbar)^3}\left\vert\varphi(\textbf{p}',t)\right\vert^2
\end{align*}

Así podemos identificar la densidad de probabilidad transformada al espacio de momentos como 

\begin{equation}
    \mathbf{W}(\textbf{p},t)=\frac{\left\vert\varphi(\textbf{p},t)\right\vert^2}{(2\pi\hbar)^3}
\end{equation}

Y definir valores esperados y desviaciones cuadráticas medias de la misma manera que en el espacio de coordenadas

\begin{align*}
    \langle \textbf{p} \rangle &= \int \frac{d^3p}{(2\pi\hbar)^3} \left\vert \varphi(\textbf{p},t) \right\vert^2 \textbf{p} \\
    (\Delta\textbf{p})^2 &=\langle(p-\langle p\rangle)^2\rangle=\int dp \mathbf{W}(p,t)(p-p_0)^2\\
\end{align*}

\subsubsection{Valor esperado del momento}
\begin{align*}
    \langle \textbf{p} \rangle &= \int \frac{d^3 p}{(2\pi\hbar)^3} \varphi(p,t)p\varphi(p,t)*= \int\int\int \frac{d^3p}{(2\pi\hbar)}d^3xd^3x'e^{\frac{i}{\hbar}p\cdot x'}\psi(x',t)*pe^{-\frac{i}{\hbar}p\cdot x}\psi(x,t)\\
    &=\int\int\int \frac{d^3p}{(2\pi\hbar)^3}d^3x d^3x'e^{\frac{i}{\hbar}p\cdot x'}\psi(x',t)*(-\frac{\hbar}{i}\nabla e^{-\frac{i}{\hbar}p\cdot x})\psi(x,t)\\
    &=\int\int d^3xd^3x'\psi(x',t)*(\frac{h}{i}\psi(x,t))\int \frac{d^3p}{(2\pi\hbar)^3}e^{\frac{i}{\hbar}p\cdot(x'-x)}
    \langle \textbf{p} \rangle &=\int d^3x \psi(x,t)*(\frac{h}{i}\nabla\psi(x,t))=\int d^3x\psi(x,t) p
\end{align*}

Así identificamos al operador momento en el espacio de las coordenadas
\begin{equation}
    p\rightarrow \frac{\hbar}{i}\nabla
\end{equation}

\section{Notación de Dirac y espacios de Hilbert}
\subsubsection{Bras, kets, operadores y producto escalar}
\textbf{Notación:} Vector en un espacio vectorial complejo de dimensión n:
\begin{multicols}{2}
    \begin{equation*}
        \vert \nu \rangle \rightarrow \begin{pmatrix}
            \nu_1\\ \nu_2\\ .\\.\\.\\\nu_N
        \end{pmatrix}
    \end{equation*}
    Las componentes $\nu_i$ son números complejos y la dimensión N del sistema depende de la naturaleza del sistema. $\vert \nu \rangle $ define el estado de un sistema.
\end{multicols}

La dimensión del espacio está relacionada con los estados que puede elegir el sistema. Si el estado se describe por una cantidad continua la dimensión del espacio es infinita.

\subsubsection{Álgebra.} Las peraciones suma y multiplicación por escalar son cerradas.
\begin{align*}
    \vert \nu \rangle+ \vert u \rangle = \vert w \rangle,  \ \alpha\vert\nu\rangle=\vert\nu\rangle\alpha=\vert w \rangle, \ 0\vert v \rangle = \vert 0 \rangle.
\end{align*}
\begin{advertencia}
    Los vectores $\vert v \rangle$ y $\alpha \vert v \rangle$ representan un mismo estado.
\end{advertencia}

\subsubsection{Observables y Operadores}
Un observable se puede representar por un operador que actúa en el espacio vectorial. Para un ket, el operador actúa por la izquierda y el resultado de su operación es otro ket.\\


\textbf{Operador A} $\mathbf{A} \vert v \rangle=\vert w \rangle$\\
Eigenket (autovector) y autovalores de un operador 

\begin{align*}
    A \vert \lambda \rangle = \lambda \vert \lambda \rangle, \ A\vert\lambda_i\rangle= \lambda_i\vert\lambda_i\rangle. \ \lambda_i \in \mathbb{C}.
\end{align*}

\begin{center}
    $\{ \lambda_i\} \rightarrow$ Conjunto de autovalores del operador A. 
\end{center}

Cualquier ket del espacio de estados de un sistema puede expresarse como una combinación de autovectores independientes (asociados a sus respectivos autovalores {$\lambda_i$}) de la forma\\

$\vert \alpha \rangle = \sum_{i=1}^{N} c_i \vert \lambda_i \rangle$, con $A \vert\lambda_i\rangle=\lambda_i\vert\lambda_i\rangle$ y $c_i \in \mathbb{C}$




\vspace*{-8mm}
\section{Dinámica de Ecuación de Schrodinger}

% \begin{advertencia}
%     Suponemos conocidas las propiedades de la función 
%     \[
%         \func{\cos}{\R}{\R}
%         \texty
%         \func{\arccos}{[-1,1]}{\R}.
%     \]
% \end{advertencia}

\begin{align*}
    &\mid \alpha \rangle = \mathcal{U}(t,t_0) \mid \alpha,t_0 \rangle\\
    &i \bar{h}\frac{\partial}{\partial t} \mathcal{U}(t,t_0) =H\mathcal{U}(t,t_0) \rightarrow i \bar{h} \frac{\partial}{\partial t}\psi
\end{align*}

$\{ \lambda_1, \lambda_2 \}$ siendo los autovalores de las formas cuadráticas $\mathbb{G}$ y $\mathbb{B}$

\begin{lem}$\lambda_1 \neq \lambda_2$ Si las direcciones principales son ortogonales.\\
    \large{\textbf{Prueba}}\\
    Buscamos, lo siguiente:
    \begin{align*}
        \langle \xi_1, \xi_2 \rangle=g_{ij}\xi_1 \xi_2=0
    \end{align*}

\end{lem}


\begin{teo}
    Sean $x\in \R$ y $y\in [-1,1]$. Se tiene que si
    \[
        y =\cos(x),
    \]
    entonces
    \[
        x = \arccos(y) + 2k\pi
        \texto
        x = -\arccos(y) + 2k\pi
    \]
    con $k\in\Z$.
\end{teo}

%%%%%%%%%%%%%%%%%%%%%%%%%%%%%%%%%%%%%%%%%%%%%%%%%%%%
%% Ejercicio 1
%%%%%%%%%%%%%%%%%%%%%%%%%%%%%%%%%%%%%%%%%%%%%%%%%%%%
\begin{ejer}
    Resolver la ecuación
    \[
        1=4\cos\left(\frac{x}{3}\right).
    \]
\end{ejer}

\begin{proof}[Solución]
    Tomemos la ecuación y dividamos entre 4, obtenemos la expresión equivalente:
    \[
        \frac{1}{4}=\cos\left(\frac{x}{3}\right),
    \]
    con lo cual, las soluciones son
    \begin{enumerate}[label=\textit{\alph*)}]
        \item\label{ej01:c01} $\displaystyle \frac{x}{3}=\arccos\left(\frac{1}{4}\right)+2k\pi$, con $k\in\Z$; o
        \item\label{ej01:c02} $\displaystyle \frac{x}{3}=-\arccos\left(\frac{1}{4}\right)+2k\pi$, con $k\in\Z$.
    \end{enumerate}
    Así, la solución de la ecuación es
    \[
        x=3\arccos\left(\frac{1}{4}\right)+6k\pi
        \qquad\text{o}\qquad
        x=-3\arccos\left(\frac{1}{4}\right)+6k\pi.\qedhere
    \]
\end{proof}


\end{document}
