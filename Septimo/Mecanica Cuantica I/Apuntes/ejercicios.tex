\documentclass{report}

\input{preamble.tex}
\newcommand{\Ab}{\mathbb{A}}	\newcommand{\Bb}{\mathbb{B}}
\newcommand{\Cb}{\mathbb{C}}	\newcommand{\Db}{\mathbb{D}}
\newcommand{\Eb}{\mathbb{E}}	\newcommand{\Fb}{\mathbb{F}}
\newcommand{\Gb}{\mathbb{G}}	\newcommand{\Hb}{\mathbb{H}}
\newcommand{\Ib}{\mathbb{I}}	\newcommand{\Jb}{\mathbb{J}}
\newcommand{\Kb}{\mathbb{K}}	\newcommand{\Lb}{\mathbb{L}}
\newcommand{\Mb}{\mathbb{M}}	\newcommand{\Nb}{\mathbb{N}}
\newcommand{\Ob}{\mathbb{O}}	\newcommand{\Pb}{\mathbb{P}}
\newcommand{\Qb}{\mathbb{Q}}	\newcommand{\Rb}{\mathbb{R}}
\newcommand{\Sb}{\mathbb{S}}	\newcommand{\Tb}{\mathbb{T}}
\newcommand{\Ub}{\mathbb{U}}	\newcommand{\Vb}{\mathbb{V}}
\newcommand{\Wb}{\mathbb{W}}	\newcommand{\Xb}{\mathbb{X}}
\newcommand{\Yb}{\mathbb{Y}}	\newcommand{\Zb}{\mathbb{Z}}
%----------------------Blackboard font--------------------
\newcommand{\cA}{\mathscr{A}}	
\newcommand{\cB}{\mathscr{B}}
\newcommand{\cC}{\mathscr{C}}
\newcommand{\cD}{\mathscr{D}}
\newcommand{\cE}{\mathscr{E}}
\newcommand{\cF}{\mathscr{F}}
\newcommand{\cG}{\mathscr{G}}
\newcommand{\cH}{\mathscr{H}}
\newcommand{\cI}{\mathscr{I}}
\newcommand{\cJ}{\mathscr{J}}
\newcommand{\cK}{\mathscr{K}}
\newcommand{\cL}{\mathscr{L}}
\newcommand{\cM}{\mathscr{M}}
\newcommand{\cN}{\mathscr{N}}
\newcommand{\cO}{\mathscr{O}}
\newcommand{\cP}{\mathscr{P}} 
\newcommand{\cQ}{\mathscr{Q}}
\newcommand{\cR}{\mathscr{R}}
\newcommand{\cS}{\mathscr{S}}
\newcommand{\cT}{\mathscr{T}}
\newcommand{\cU}{\mathscr{U}}
\newcommand{\cV}{\mathscr{V}}
\newcommand{\cW}{\mathscr{W}}
\newcommand{\cX}{\mathscr{X}}
\newcommand{\cY}{\mathscr{Y}}
\newcommand{\cZ}{\mathscr{Z}}
%-----------------------Caligraphic font--------------------
%Captital Letters
\newcommand{\bmA}{\boldsymbol{A}}	\newcommand{\bmB}{\boldsymbol{B}}
\newcommand{\bmC}{\boldsymbol{C}}	\newcommand{\bmD}{\boldsymbol{D}}
\newcommand{\bmE}{\boldsymbol{E}}	\newcommand{\bmF}{\boldsymbol{F}}
\newcommand{\bmG}{\boldsymbol{G}}	\newcommand{\bmH}{\boldsymbol{H}}
\newcommand{\bmI}{\boldsymbol{I}}	\newcommand{\bmJ}{\boldsymbol{J}}
\newcommand{\bmK}{\boldsymbol{K}}	\newcommand{\bmL}{\boldsymbol{L}}
\newcommand{\bmM}{\boldsymbol{M}}	\newcommand{\bmN}{\boldsymbol{N}}
\newcommand{\bmO}{\boldsymbol{O}}	\newcommand{\bmP}{\boldsymbol{P}}
\newcommand{\bmQ}{\boldsymbol{Q}}	\newcommand{\bmR}{\boldsymbol{R}}
\newcommand{\bmS}{\boldsymbol{S}}	\newcommand{\bmT}{\boldsymbol{T}}
\newcommand{\bmU}{\boldsymbol{U}}	\newcommand{\bmV}{\boldsymbol{V}}
\newcommand{\bmW}{\boldsymbol{W}}	\newcommand{\bmX}{\boldsymbol{X}}
\newcommand{\bmY}{\boldsymbol{Y}}	\newcommand{\bmZ}{\boldsymbol{Z}}
%Small Letters
\newcommand{\bma}{\boldsymbol{a}}	\newcommand{\bmb}{\boldsymbol{b}}
\newcommand{\bmc}{\boldsymbol{c}}	\newcommand{\bmd}{\boldsymbol{d}}
\newcommand{\bme}{\boldsymbol{e}}	\newcommand{\bmf}{\boldsymbol{f}}
\newcommand{\bmg}{\boldsymbol{g}}	\newcommand{\bmh}{\boldsymbol{h}}
\newcommand{\bmi}{\boldsymbol{i}}	\newcommand{\bmj}{\boldsymbol{j}}
\newcommand{\bmk}{\boldsymbol{k}}	\newcommand{\bml}{\boldsymbol{l}}
\newcommand{\bmm}{\boldsymbol{m}}	\newcommand{\bmn}{\boldsymbol{n}}
\newcommand{\bmo}{\boldsymbol{o}}	\newcommand{\bmp}{\boldsymbol{p}}
\newcommand{\bmq}{\boldsymbol{q}}	\newcommand{\bmr}{\boldsymbol{r}}
\newcommand{\bms}{\boldsymbol{s}}	\newcommand{\bmt}{\boldsymbol{t}}
\newcommand{\bmu}{\boldsymbol{u}}	\newcommand{\bmv}{\boldsymbol{v}}
\newcommand{\bmw}{\boldsymbol{w}}	\newcommand{\bmx}{\boldsymbol{x}}
\newcommand{\bmy}{\boldsymbol{y}}	\newcommand{\bmz}{\boldsymbol{z}}

\newcommand{\inv}{^{-1}}
\newcommand{\defi}{\equiv}
\newcommand{\gc}{^\circ}
\newcommand{\ii}{\item}
\newcommand{\ssi}{\leftrightarrow}
\newcommand{\sie}{\rightarrow}
\newcommand{\opname}{\operatorname}
\DeclareMathOperator{\img}{Im}
\DeclareMathOperator{\dom}{Dom} 
\DeclareMathOperator{\tr}{Tr}
\newcommand{\eps}{\epsilon}


\title{\Huge{Mécanica Cuántica}\\Ejercicios}
\author{\huge{Angel Almonacid}}
\date{\today}

\begin{document}
  \maketitle
  \qs{Schwabl 2.3}{
    Using the Bohr–Sommerfeld quantization rules, determine the energy eigen-
states of a particle of mass m moving in an infinitely high potential well:
\[
  V(x)=\left\lbrace \begin{array}{lcc} 0 & 0 \leq x \leq a \\ \infty & \text{otherwise} \end{array}
\] 
\large{\textbf{Respuesta.}} Fuera de la barrera de potencial la función de onda es cero, dentro el potencial vale 0, cumpliendo la ecuación diferencial
\[
 -\frac{\hbar ^2}{2m}\frac{d ^2 \psi}{d x ^2}=E\psi
\]
Escribiéndola de manera alterna tenemos 
\begin{align*}
  \frac{d ^2 \psi }{d x ^2}=-k ^2 \psi, & \ \text{ donde } k \equiv \frac{\sqrt{2mE}}{\hbar}
\end{align*}
Así identificamos la solución para un oscilador armónica
\[
  \psi(x)=A\sin{kx}+B\cos{kx}
\]
y aplicando las siguientes condiciones iniciales 
\[
  \psi(0)=\psi(a)=0.
\]
Encontramos:
\begin{align*}
  \psi(0)&=A\sin{0}+B\cos{0}=B=0. \\ 
  \psi(x)&=A\sin{kx}\\
  \psi(a)&=A\sin{ka}=0 \rightarrow \sin{ka}=0\\
         &ka=n\pi\\
         &k=\frac{n\pi}{a}
\end{align*}
Sabiendo que el exponente se comporta indicialmente, la energía también tiene este comportamiento y está asociada de la siguiente manera:
\[
  E_n=\frac{h ^2 k_n ^2}{2m}= \frac{n ^2 \pi ^2 \hbar ^2}{2ma ^2}
\]
}

\qs{Schwabl 2.4}{
  \begin{enumerate}
    \ii $p^\dagger=p$, como $\bmp$ es un operador, debe cumplir: $\bra{\psi}\bmp \ket{\phi}=\bra{\phi} \bmp \ket{\psi}^*$ 
    \begin{align*}
      \bra{\psi}\bmp \ket{\phi}&=\int \phi ^* \left( x \psi \right) dx\\
      \bra{\phi} \bmp \ket{\psi}^*&=\int \left(\phi \bmp \psi^* \right)^* dx\\
             &=\int \left( \phi -i \hbar \frac{\partial \psi^* }{\partial x})^*dx\\
             &=i \hbar \int \phi^*\frac{\partial \psi  }{\partial x}  dx\\
      \text{Integrando por partes}:\\
             &=\left[\phi^* \psi \right]_{-\infty}^{\infty}-i \hbar \int \psi^* \frac{\partial \phi  }{\partial x} dx\\
             &=\int \phi^* \left(-i \hbar  \frac{\partial  }{\partial x}\right)\psi dx\\
             &=\int \phi ^* \bmp \psi dx=\bra{\psi}\bmp \ket{\phi}\qed
    \end{align*}
    \ii $\left(AB \right)^\dagger=B ^\dagger A ^\dagger$. Desarrollando el producto:\\
    \begin{align*}
      A=\begin{bmatrix}
          a11 & a12 & \dots & a1n \\
          a21 & a22 & \dots & a2n \\
          \dots & \dots & \dots & \dots\\
          an1 & an2 & \dots & ann
      \end{bmatrix}    
    &&B=\begin{bmatrix}
        b11 & b12 & \dots & b1n \\
        b21 & b22 & \dots & b2n \\
        \dots & \dots & \dots & \dots \\
        bn1 & bn2 & \dots & bnn
    \end{bmatrix}  
        \end{align*}
        \begin{align*}
          A \cdot B= \begin{bmatrix}
        a11b11+...+a1nbn1 & a11b12+...+a1nb2n & \dots & a11b1n+...+a1nbnn \\
        a21b11+...+a2nbn1 & a21b12+...+a2nb2n & \dots & a21b1n+...+a2nbnn \\
        \dots & \dots & \dots & \dots \\
       an1b11+...+annbn1 & an1b12+...+annb2n & \dots & an1b1n+...+annbnn
    \end{bmatrix}  \\   
    (A \cdot B)^\dagger= \begin{bmatrix}
        a11b11+...+a1nbn1 & a21b11+...+a2nbn1 & \dots & an1b11+...+annbn1 \\
        a11b12+...+a1nb2n & a21b12+...+a2nb2n & \dots & an1b12+...+annb2n \\
        \dots & \dots & \dots & \dots \\
       a11b1n+...+a1nbnn & a21b1n+...+a2nbnn & \dots & an1b1n+...+annbnn
    \end{bmatrix}^*
        \end{align*}
  \end{enumerate}
}
\end{document}
