\documentclass{report}
\input{preamble}
\newcommand{\inv}{^{-1}}
\newcommand{\defi}{\equiv}
\newcommand{\gc}{^\circ}
\newcommand{\ii}{\item}
\newcommand{\ssi}{\leftrightarrow}
\newcommand{\sie}{\rightarrow}
\newcommand{\opname}{\operatorname}
\DeclareMathOperator{\img}{Im}
\DeclareMathOperator{\dom}{Dom} 
\DeclareMathOperator{\tr}{Tr}
\newcommand{\eps}{\epsilon}

\newcommand{\Ab}{\mathbb{A}}	\newcommand{\Bb}{\mathbb{B}}
\newcommand{\Cb}{\mathbb{C}}	\newcommand{\Db}{\mathbb{D}}
\newcommand{\Eb}{\mathbb{E}}	\newcommand{\Fb}{\mathbb{F}}
\newcommand{\Gb}{\mathbb{G}}	\newcommand{\Hb}{\mathbb{H}}
\newcommand{\Ib}{\mathbb{I}}	\newcommand{\Jb}{\mathbb{J}}
\newcommand{\Kb}{\mathbb{K}}	\newcommand{\Lb}{\mathbb{L}}
\newcommand{\Mb}{\mathbb{M}}	\newcommand{\Nb}{\mathbb{N}}
\newcommand{\Ob}{\mathbb{O}}	\newcommand{\Pb}{\mathbb{P}}
\newcommand{\Qb}{\mathbb{Q}}	\newcommand{\Rb}{\mathbb{R}}
\newcommand{\Sb}{\mathbb{S}}	\newcommand{\Tb}{\mathbb{T}}
\newcommand{\Ub}{\mathbb{U}}	\newcommand{\Vb}{\mathbb{V}}
\newcommand{\Wb}{\mathbb{W}}	\newcommand{\Xb}{\mathbb{X}}
\newcommand{\Yb}{\mathbb{Y}}	\newcommand{\Zb}{\mathbb{Z}}
%----------------------Blackboard font--------------------
\newcommand{\cA}{\mathscr{A}}	
\newcommand{\cB}{\mathscr{B}}
\newcommand{\cC}{\mathscr{C}}
\newcommand{\cD}{\mathscr{D}}
\newcommand{\cE}{\mathscr{E}}
\newcommand{\cF}{\mathscr{F}}
\newcommand{\cG}{\mathscr{G}}
\newcommand{\cH}{\mathscr{H}}
\newcommand{\cI}{\mathscr{I}}
\newcommand{\cJ}{\mathscr{J}}
\newcommand{\cK}{\mathscr{K}}
\newcommand{\cL}{\mathscr{L}}
\newcommand{\cM}{\mathscr{M}}
\newcommand{\cN}{\mathscr{N}}
\newcommand{\cO}{\mathscr{O}}
\newcommand{\cP}{\mathscr{P}} 
\newcommand{\cQ}{\mathscr{Q}}
\newcommand{\cR}{\mathscr{R}}
\newcommand{\cS}{\mathscr{S}}
\newcommand{\cT}{\mathscr{T}}
\newcommand{\cU}{\mathscr{U}}
\newcommand{\cV}{\mathscr{V}}
\newcommand{\cW}{\mathscr{W}}
\newcommand{\cX}{\mathscr{X}}
\newcommand{\cY}{\mathscr{Y}}
\newcommand{\cZ}{\mathscr{Z}}
%-----------------------Caligraphic font--------------------
%Captital Letters
\newcommand{\bmA}{\boldsymbol{A}}	\newcommand{\bmB}{\boldsymbol{B}}
\newcommand{\bmC}{\boldsymbol{C}}	\newcommand{\bmD}{\boldsymbol{D}}
\newcommand{\bmE}{\boldsymbol{E}}	\newcommand{\bmF}{\boldsymbol{F}}
\newcommand{\bmG}{\boldsymbol{G}}	\newcommand{\bmH}{\boldsymbol{H}}
\newcommand{\bmI}{\boldsymbol{I}}	\newcommand{\bmJ}{\boldsymbol{J}}
\newcommand{\bmK}{\boldsymbol{K}}	\newcommand{\bmL}{\boldsymbol{L}}
\newcommand{\bmM}{\boldsymbol{M}}	\newcommand{\bmN}{\boldsymbol{N}}
\newcommand{\bmO}{\boldsymbol{O}}	\newcommand{\bmP}{\boldsymbol{P}}
\newcommand{\bmQ}{\boldsymbol{Q}}	\newcommand{\bmR}{\boldsymbol{R}}
\newcommand{\bmS}{\boldsymbol{S}}	\newcommand{\bmT}{\boldsymbol{T}}
\newcommand{\bmU}{\boldsymbol{U}}	\newcommand{\bmV}{\boldsymbol{V}}
\newcommand{\bmW}{\boldsymbol{W}}	\newcommand{\bmX}{\boldsymbol{X}}
\newcommand{\bmY}{\boldsymbol{Y}}	\newcommand{\bmZ}{\boldsymbol{Z}}
%Small Letters
\newcommand{\bma}{\boldsymbol{a}}	\newcommand{\bmb}{\boldsymbol{b}}
\newcommand{\bmc}{\boldsymbol{c}}	\newcommand{\bmd}{\boldsymbol{d}}
\newcommand{\bme}{\boldsymbol{e}}	\newcommand{\bmf}{\boldsymbol{f}}
\newcommand{\bmg}{\boldsymbol{g}}	\newcommand{\bmh}{\boldsymbol{h}}
\newcommand{\bmi}{\boldsymbol{i}}	\newcommand{\bmj}{\boldsymbol{j}}
\newcommand{\bmk}{\boldsymbol{k}}	\newcommand{\bml}{\boldsymbol{l}}
\newcommand{\bmm}{\boldsymbol{m}}	\newcommand{\bmn}{\boldsymbol{n}}
\newcommand{\bmo}{\boldsymbol{o}}	\newcommand{\bmp}{\boldsymbol{p}}
\newcommand{\bmq}{\boldsymbol{q}}	\newcommand{\bmr}{\boldsymbol{r}}
\newcommand{\bms}{\boldsymbol{s}}	\newcommand{\bmt}{\boldsymbol{t}}
\newcommand{\bmu}{\boldsymbol{u}}	\newcommand{\bmv}{\boldsymbol{v}}
\newcommand{\bmw}{\boldsymbol{w}}	\newcommand{\bmx}{\boldsymbol{x}}
\newcommand{\bmy}{\boldsymbol{y}}	\newcommand{\bmz}{\boldsymbol{z}}

\title{\Huge{Métodos Geométricos}\\Apuntes}
\author{\huge{Angel Almonacid}}
\date{\today}
\begin{document}
  \maketitle
  \chapter{Variedades y Tensores}
  \section{Variedades diferenciables}
  \dfn{Variedad}{
    Una variedad n-dimensional es un conjunto de M puntos y una familia de mapeos
    y vecindarios ${(U_i,\phi_i)}$, tal que la familia ${U_i}$ cubre la variedad y $\phi_i$ es un homeomorfismo de $U_i$  a un subconjunto abierto $U_i^{'}\in \Rb$, que a su vez cumplen las siguientes condiciones:
    begin{\begin{itemize} 
      \ii $\cup_i U_i=M$
      \ii $\forall i,j, \cup_i \cap \cup_j\neq\emptyset=W$, $\phi_i(W),\phi_j(W)$ son conjuntos abiertos en $\Rb^n$ y los mapeos $\psi_ij\equiv\phi_i\circ\phi_j^-1$
    
    \end{itemize}
  }
  }
  \textbf{Notacion:}
  \begin{itemize}
    \ii $U_i \rightarrow$ vecindario
    \ii $(U_i,\phi_i)\rightarrow$ carta
    \ii ${(U_i, \phi_i)} \rightarrow$ atlas, define estructura deiferenciable en M.
  \end{itemize}

  
  \subsection{Mapas entre variedades}
  Para dos variedades M y N mapeadas por $\psi(p)={x^\mu}, \phi(f(p))={y^\mu}$
  \begin{gather*}
    f: M \rightarrow N\\
    y=\psi \circ f \circ \phi^-1(x)\\
    \text{Podemos tomar como notación}\rightarrow y=f(x), y^\mu=f(x^\mu)
  \end{gather*}
  Dos casos importantes de mapas entre variedades son las curvas y las funciones.
  \subsection*{Curvas}
  \[
    c: (a,b)\rightarrow M. a<0<b
  \]
  En una carta $(U,\phi)$, la curva tiene la representación en coordenadas:
  \[
    x=\phi \circ c : \Rb \rightarrow \Rb^m
  \]
  \subsection*{Funciones}
  \[
    f: M \rightarrow \Rb
  \]
  En una carta $(U,\phi)$ la función $f$ tiene la representación en coordenadas:
  \[
    f \circ \phi ^{-1}:\Rb^m\rightarrow\Rb
  \]
  Al conjunto de funciones suaves en M lo llamaremos $\cF(x)$.
    \subsection{Vector Tangente}
    \textbf{Vector:} Para definir el vector tangente tomaremos una curva y una función definida en M:
    \[
      f(c(t)):(a,b)\rightarrow \Rb
    \]
  Definiendo la derivada de f a lo largo de C, tenemos:
  \begin{gather*}
    \frac{d f(c(t )) }{d t}_{t=0}=\frac{d x^i(c(t)) }{d t}_{t=0} \frac{\partial f }{\partial x^i}
  \end{gather*}
\end{document}
