\documentclass{report}
%%%%%%%%%%%%%%%%%%%%%%%%%%%%%%%%%
% PACKAGE IMPORTS
%%%%%%%%%%%%%%%%%%%%%%%%%%%%%%%%%


\usepackage[tmargin=2cm,rmargin=1in,lmargin=1in,margin=0.85in,bmargin=2cm,footskip=.2in]{geometry}
\usepackage{amsmath,amsfonts,amsthm,amssymb,mathtools}
\usepackage[varbb]{newpxmath}
\usepackage{xfrac}
\usepackage[makeroom]{cancel}
\usepackage{mathtools}
\usepackage{bookmark}
\usepackage{enumitem}
\usepackage{hyperref,theoremref}
\hypersetup{
	pdftitle={Assignment},
	colorlinks=true, linkcolor=doc!90,
	bookmarksnumbered=true,
	bookmarksopen=true
}
\usepackage[most,many,breakable]{tcolorbox}
\usepackage{xcolor}
\usepackage{varwidth}
\usepackage{varwidth}
\usepackage{etoolbox}
%\usepackage{authblk}
\usepackage{nameref}
\usepackage{multicol,array}
\usepackage{tikz-cd}
\usepackage[ruled,vlined,linesnumbered]{algorithm2e}
\usepackage{comment} % enables the use of multi-line comments (\ifx \fi) 
\usepackage{import}
\usepackage{xifthen}
\usepackage{pdfpages}
\usepackage{transparent}
\usepackage{braket}

\newcommand\mycommfont[1]{\footnotesize\ttfamily\textcolor{blue}{#1}}
\SetCommentSty{mycommfont}
\newcommand{\incfig}[1]{%
    \def\svgwidth{\columnwidth}
    \import{./figures/}{#1.pdf_tex}
}

\usepackage{tikzsymbols}
\renewcommand\qedsymbol{$\Laughey$}


%\usepackage{import}
%\usepackage{xifthen}
%\usepackage{pdfpages}
%\usepackage{transparent}


%%%%%%%%%%%%%%%%%%%%%%%%%%%%%%
% SELF MADE COLORS
%%%%%%%%%%%%%%%%%%%%%%%%%%%%%%



\definecolor{myg}{RGB}{56, 140, 70}
\definecolor{myb}{RGB}{45, 111, 177}
\definecolor{myr}{RGB}{199, 68, 64}
\definecolor{mytheorembg}{HTML}{F2F2F9}
\definecolor{mytheoremfr}{HTML}{00007B}
\definecolor{mylenmabg}{HTML}{FFFAF8}
\definecolor{mylenmafr}{HTML}{983b0f}
\definecolor{mypropbg}{HTML}{f2fbfc}
\definecolor{mypropfr}{HTML}{191971}
\definecolor{myexamplebg}{HTML}{F2FBF8}
\definecolor{myexamplefr}{HTML}{88D6D1}
\definecolor{myexampleti}{HTML}{2A7F7F}
\definecolor{mydefinitbg}{HTML}{E5E5FF}
\definecolor{mydefinitfr}{HTML}{3F3FA3}
\definecolor{notesgreen}{RGB}{0,162,0}
\definecolor{myp}{RGB}{197, 92, 212}
\definecolor{mygr}{HTML}{2C3338}
\definecolor{myred}{RGB}{127,0,0}
\definecolor{myyellow}{RGB}{169,121,69}
\definecolor{myexercisebg}{HTML}{F2FBF8}
\definecolor{myexercisefg}{HTML}{88D6D1}


%%%%%%%%%%%%%%%%%%%%%%%%%%%%
% TCOLORBOX SETUPS
%%%%%%%%%%%%%%%%%%%%%%%%%%%%

\setlength{\parindent}{1cm}
%================================
% THEOREM BOX
%================================

\tcbuselibrary{theorems,skins,hooks}
\newtcbtheorem[number within=section]{Theorem}{Theorem}
{%
	enhanced,
	breakable,
	colback = mytheorembg,
	frame hidden,
	boxrule = 0sp,
	borderline west = {2pt}{0pt}{mytheoremfr},
	sharp corners,
	detach title,
	before upper = \tcbtitle\par\smallskip,
	coltitle = mytheoremfr,
	fonttitle = \bfseries\sffamily,
	description font = \mdseries,
	separator sign none,
	segmentation style={solid, mytheoremfr},
}
{th}

\tcbuselibrary{theorems,skins,hooks}
\newtcbtheorem[number within=chapter]{theorem}{Theorem}
{%
	enhanced,
	breakable,
	colback = mytheorembg,
	frame hidden,
	boxrule = 0sp,
	borderline west = {2pt}{0pt}{mytheoremfr},
	sharp corners,
	detach title,
	before upper = \tcbtitle\par\smallskip,
	coltitle = mytheoremfr,
	fonttitle = \bfseries\sffamily,
	description font = \mdseries,
	separator sign none,
	segmentation style={solid, mytheoremfr},
}
{th}


\tcbuselibrary{theorems,skins,hooks}
\newtcolorbox{Theoremcon}
{%
	enhanced
	,breakable
	,colback = mytheorembg
	,frame hidden
	,boxrule = 0sp
	,borderline west = {2pt}{0pt}{mytheoremfr}
	,sharp corners
	,description font = \mdseries
	,separator sign none
}

%================================
% Corollery
%================================
\tcbuselibrary{theorems,skins,hooks}
\newtcbtheorem[number within=section]{Corollary}{Corollary}
{%
	enhanced
	,breakable
	,colback = myp!10
	,frame hidden
	,boxrule = 0sp
	,borderline west = {2pt}{0pt}{myp!85!black}
	,sharp corners
	,detach title
	,before upper = \tcbtitle\par\smallskip
	,coltitle = myp!85!black
	,fonttitle = \bfseries\sffamily
	,description font = \mdseries
	,separator sign none
	,segmentation style={solid, myp!85!black}
}
{th}
\tcbuselibrary{theorems,skins,hooks}
\newtcbtheorem[number within=chapter]{corollary}{Corollary}
{%
	enhanced
	,breakable
	,colback = myp!10
	,frame hidden
	,boxrule = 0sp
	,borderline west = {2pt}{0pt}{myp!85!black}
	,sharp corners
	,detach title
	,before upper = \tcbtitle\par\smallskip
	,coltitle = myp!85!black
	,fonttitle = \bfseries\sffamily
	,description font = \mdseries
	,separator sign none
	,segmentation style={solid, myp!85!black}
}
{th}


%================================
% LENMA
%================================

\tcbuselibrary{theorems,skins,hooks}
\newtcbtheorem[number within=section]{Lenma}{Lenma}
{%
	enhanced,
	breakable,
	colback = mylenmabg,
	frame hidden,
	boxrule = 0sp,
	borderline west = {2pt}{0pt}{mylenmafr},
	sharp corners,
	detach title,
	before upper = \tcbtitle\par\smallskip,
	coltitle = mylenmafr,
	fonttitle = \bfseries\sffamily,
	description font = \mdseries,
	separator sign none,
	segmentation style={solid, mylenmafr},
}
{th}

\tcbuselibrary{theorems,skins,hooks}
\newtcbtheorem[number within=chapter]{lenma}{Lenma}
{%
	enhanced,
	breakable,
	colback = mylenmabg,
	frame hidden,
	boxrule = 0sp,
	borderline west = {2pt}{0pt}{mylenmafr},
	sharp corners,
	detach title,
	before upper = \tcbtitle\par\smallskip,
	coltitle = mylenmafr,
	fonttitle = \bfseries\sffamily,
	description font = \mdseries,
	separator sign none,
	segmentation style={solid, mylenmafr},
}
{th}


%================================
% PROPOSITION
%================================

\tcbuselibrary{theorems,skins,hooks}
\newtcbtheorem[number within=section]{Prop}{Proposition}
{%
	enhanced,
	breakable,
	colback = mypropbg,
	frame hidden,
	boxrule = 0sp,
	borderline west = {2pt}{0pt}{mypropfr},
	sharp corners,
	detach title,
	before upper = \tcbtitle\par\smallskip,
	coltitle = mypropfr,
	fonttitle = \bfseries\sffamily,
	description font = \mdseries,
	separator sign none,
	segmentation style={solid, mypropfr},
}
{th}

\tcbuselibrary{theorems,skins,hooks}
\newtcbtheorem[number within=chapter]{prop}{Proposition}
{%
	enhanced,
	breakable,
	colback = mypropbg,
	frame hidden,
	boxrule = 0sp,
	borderline west = {2pt}{0pt}{mypropfr},
	sharp corners,
	detach title,
	before upper = \tcbtitle\par\smallskip,
	coltitle = mypropfr,
	fonttitle = \bfseries\sffamily,
	description font = \mdseries,
	separator sign none,
	segmentation style={solid, mypropfr},
}
{th}


%================================
% CLAIM
%================================

\tcbuselibrary{theorems,skins,hooks}
\newtcbtheorem[number within=section]{claim}{Claim}
{%
	enhanced
	,breakable
	,colback = myg!10
	,frame hidden
	,boxrule = 0sp
	,borderline west = {2pt}{0pt}{myg}
	,sharp corners
	,detach title
	,before upper = \tcbtitle\par\smallskip
	,coltitle = myg!85!black
	,fonttitle = \bfseries\sffamily
	,description font = \mdseries
	,separator sign none
	,segmentation style={solid, myg!85!black}
}
{th}



%================================
% Exercise
%================================

\tcbuselibrary{theorems,skins,hooks}
\newtcbtheorem[number within=section]{Exercise}{Exercise}
{%
	enhanced,
	breakable,
	colback = myexercisebg,
	frame hidden,
	boxrule = 0sp,
	borderline west = {2pt}{0pt}{myexercisefg},
	sharp corners,
	detach title,
	before upper = \tcbtitle\par\smallskip,
	coltitle = myexercisefg,
	fonttitle = \bfseries\sffamily,
	description font = \mdseries,
	separator sign none,
	segmentation style={solid, myexercisefg},
}
{th}

\tcbuselibrary{theorems,skins,hooks}
\newtcbtheorem[number within=chapter]{exercise}{Exercise}
{%
	enhanced,
	breakable,
	colback = myexercisebg,
	frame hidden,
	boxrule = 0sp,
	borderline west = {2pt}{0pt}{myexercisefg},
	sharp corners,
	detach title,
	before upper = \tcbtitle\par\smallskip,
	coltitle = myexercisefg,
	fonttitle = \bfseries\sffamily,
	description font = \mdseries,
	separator sign none,
	segmentation style={solid, myexercisefg},
}
{th}

%================================
% EXAMPLE BOX
%================================

\newtcbtheorem[number within=section]{Example}{Example}
{%
	colback = myexamplebg
	,breakable
	,colframe = myexamplefr
	,coltitle = myexampleti
	,boxrule = 1pt
	,sharp corners
	,detach title
	,before upper=\tcbtitle\par\smallskip
	,fonttitle = \bfseries
	,description font = \mdseries
	,separator sign none
	,description delimiters parenthesis
}
{ex}

\newtcbtheorem[number within=chapter]{example}{Example}
{%
	colback = myexamplebg
	,breakable
	,colframe = myexamplefr
	,coltitle = myexampleti
	,boxrule = 1pt
	,sharp corners
	,detach title
	,before upper=\tcbtitle\par\smallskip
	,fonttitle = \bfseries
	,description font = \mdseries
	,separator sign none
	,description delimiters parenthesis
}
{ex}

%================================
% DEFINITION BOX
%================================

\newtcbtheorem[number within=section]{Definition}{Definition}{enhanced,
	before skip=2mm,after skip=2mm, colback=red!5,colframe=red!80!black,boxrule=0.5mm,
	attach boxed title to top left={xshift=1cm,yshift*=1mm-\tcboxedtitleheight}, varwidth boxed title*=-3cm,
	boxed title style={frame code={
					\path[fill=tcbcolback]
					([yshift=-1mm,xshift=-1mm]frame.north west)
					arc[start angle=0,end angle=180,radius=1mm]
					([yshift=-1mm,xshift=1mm]frame.north east)
					arc[start angle=180,end angle=0,radius=1mm];
					\path[left color=tcbcolback!60!black,right color=tcbcolback!60!black,
						middle color=tcbcolback!80!black]
					([xshift=-2mm]frame.north west) -- ([xshift=2mm]frame.north east)
					[rounded corners=1mm]-- ([xshift=1mm,yshift=-1mm]frame.north east)
					-- (frame.south east) -- (frame.south west)
					-- ([xshift=-1mm,yshift=-1mm]frame.north west)
					[sharp corners]-- cycle;
				},interior engine=empty,
		},
	fonttitle=\bfseries,
	title={#2},#1}{def}
\newtcbtheorem[number within=chapter]{definition}{Definition}{enhanced,
	before skip=2mm,after skip=2mm, colback=red!5,colframe=red!80!black,boxrule=0.5mm,
	attach boxed title to top left={xshift=1cm,yshift*=1mm-\tcboxedtitleheight}, varwidth boxed title*=-3cm,
	boxed title style={frame code={
					\path[fill=tcbcolback]
					([yshift=-1mm,xshift=-1mm]frame.north west)
					arc[start angle=0,end angle=180,radius=1mm]
					([yshift=-1mm,xshift=1mm]frame.north east)
					arc[start angle=180,end angle=0,radius=1mm];
					\path[left color=tcbcolback!60!black,right color=tcbcolback!60!black,
						middle color=tcbcolback!80!black]
					([xshift=-2mm]frame.north west) -- ([xshift=2mm]frame.north east)
					[rounded corners=1mm]-- ([xshift=1mm,yshift=-1mm]frame.north east)
					-- (frame.south east) -- (frame.south west)
					-- ([xshift=-1mm,yshift=-1mm]frame.north west)
					[sharp corners]-- cycle;
				},interior engine=empty,
		},
	fonttitle=\bfseries,
	title={#2},#1}{def}



%================================
% Solution BOX
%================================

\makeatletter
\newtcbtheorem{question}{Question}{enhanced,
	breakable,
	colback=white,
	colframe=myb!80!black,
	attach boxed title to top left={yshift*=-\tcboxedtitleheight},
	fonttitle=\bfseries,
	title={#2},
	boxed title size=title,
	boxed title style={%
			sharp corners,
			rounded corners=northwest,
			colback=tcbcolframe,
			boxrule=0pt,
		},
	underlay boxed title={%
			\path[fill=tcbcolframe] (title.south west)--(title.south east)
			to[out=0, in=180] ([xshift=5mm]title.east)--
			(title.center-|frame.east)
			[rounded corners=\kvtcb@arc] |-
			(frame.north) -| cycle;
		},
	#1
}{def}
\makeatother

%================================
% SOLUTION BOX
%================================

\makeatletter
\newtcolorbox{solution}{enhanced,
	breakable,
	colback=white,
	colframe=myg!80!black,
	attach boxed title to top left={yshift*=-\tcboxedtitleheight},
	title=Solution,
	boxed title size=title,
	boxed title style={%
			sharp corners,
			rounded corners=northwest,
			colback=tcbcolframe,
			boxrule=0pt,
		},
	underlay boxed title={%
			\path[fill=tcbcolframe] (title.south west)--(title.south east)
			to[out=0, in=180] ([xshift=5mm]title.east)--
			(title.center-|frame.east)
			[rounded corners=\kvtcb@arc] |-
			(frame.north) -| cycle;
		},
}
\makeatother

%================================
% Question BOX
%================================

\makeatletter
\newtcbtheorem{qstion}{Question}{enhanced,
	breakable,
	colback=white,
	colframe=mygr,
	attach boxed title to top left={yshift*=-\tcboxedtitleheight},
	fonttitle=\bfseries,
	title={#2},
	boxed title size=title,
	boxed title style={%
			sharp corners,
			rounded corners=northwest,
			colback=tcbcolframe,
			boxrule=0pt,
		},
	underlay boxed title={%
			\path[fill=tcbcolframe] (title.south west)--(title.south east)
			to[out=0, in=180] ([xshift=5mm]title.east)--
			(title.center-|frame.east)
			[rounded corners=\kvtcb@arc] |-
			(frame.north) -| cycle;
		},
	#1
}{def}
\makeatother

\newtcbtheorem[number within=chapter]{wconc}{Wrong Concept}{
	breakable,
	enhanced,
	colback=white,
	colframe=myr,
	arc=0pt,
	outer arc=0pt,
	fonttitle=\bfseries\sffamily\large,
	colbacktitle=myr,
	attach boxed title to top left={},
	boxed title style={
			enhanced,
			skin=enhancedfirst jigsaw,
			arc=3pt,
			bottom=0pt,
			interior style={fill=myr}
		},
	#1
}{def}



%================================
% NOTE BOX
%================================

\usetikzlibrary{arrows,calc,shadows.blur}
\tcbuselibrary{skins}
\newtcolorbox{note}[1][]{%
	enhanced jigsaw,
	colback=gray!20!white,%
	colframe=gray!80!black,
	size=small,
	boxrule=1pt,
	title=\textbf{Note:-},
	halign title=flush center,
	coltitle=black,
	breakable,
	drop shadow=black!50!white,
	attach boxed title to top left={xshift=1cm,yshift=-\tcboxedtitleheight/2,yshifttext=-\tcboxedtitleheight/2},
	minipage boxed title=1.5cm,
	boxed title style={%
			colback=white,
			size=fbox,
			boxrule=1pt,
			boxsep=2pt,
			underlay={%
					\coordinate (dotA) at ($(interior.west) + (-0.5pt,0)$);
					\coordinate (dotB) at ($(interior.east) + (0.5pt,0)$);
					\begin{scope}
						\clip (interior.north west) rectangle ([xshift=3ex]interior.east);
						\filldraw [white, blur shadow={shadow opacity=60, shadow yshift=-.75ex}, rounded corners=2pt] (interior.north west) rectangle (interior.south east);
					\end{scope}
					\begin{scope}[gray!80!black]
						\fill (dotA) circle (2pt);
						\fill (dotB) circle (2pt);
					\end{scope}
				},
		},
	#1,
}

%%%%%%%%%%%%%%%%%%%%%%%%%%%%%%
% SELF MADE COMMANDS
%%%%%%%%%%%%%%%%%%%%%%%%%%%%%%


\newcommand{\thm}[2]{\begin{Theorem}{#1}{}#2\end{Theorem}}
\newcommand{\cor}[2]{\begin{Corollary}{#1}{}#2\end{Corollary}}
\newcommand{\mlenma}[2]{\begin{Lenma}{#1}{}#2\end{Lenma}}
\newcommand{\mprop}[2]{\begin{Prop}{#1}{}#2\end{Prop}}
\newcommand{\clm}[3]{\begin{claim}{#1}{#2}#3\end{claim}}
\newcommand{\wc}[2]{\begin{wconc}{#1}{}\setlength{\parindent}{1cm}#2\end{wconc}}
\newcommand{\thmcon}[1]{\begin{Theoremcon}{#1}\end{Theoremcon}}
\newcommand{\ex}[2]{\begin{Example}{#1}{}#2\end{Example}}
\newcommand{\dfn}[2]{\begin{Definition}[colbacktitle=red!75!black]{#1}{}#2\end{Definition}}
\newcommand{\dfnc}[2]{\begin{definition}[colbacktitle=red!75!black]{#1}{}#2\end{definition}}
\newcommand{\qs}[2]{\begin{question}{#1}{}#2\end{question}}
\newcommand{\pf}[2]{\begin{myproof}[#1]#2\end{myproof}}
\newcommand{\nt}[1]{\begin{note}#1\end{note}}

\newcommand*\circled[1]{\tikz[baseline=(char.base)]{
		\node[shape=circle,draw,inner sep=1pt] (char) {#1};}}
\newcommand\getcurrentref[1]{%
	\ifnumequal{\value{#1}}{0}
	{??}
	{\the\value{#1}}%
}
\newcommand{\getCurrentSectionNumber}{\getcurrentref{section}}
\newenvironment{myproof}[1][\proofname]{%
	\proof[\bfseries #1: ]%
}{\endproof}

\newcommand{\mclm}[2]{\begin{myclaim}[#1]#2\end{myclaim}}
\newenvironment{myclaim}[1][\claimname]{\proof[\bfseries #1: ]}{}

\newcounter{mylabelcounter}

\makeatletter
\newcommand{\setword}[2]{%
	\phantomsection
	#1\def\@currentlabel{\unexpanded{#1}}\label{#2}%
}
\makeatother




\tikzset{
	symbol/.style={
			draw=none,
			every to/.append style={
					edge node={node [sloped, allow upside down, auto=false]{$#1$}}}
		}
}


% deliminators
\DeclarePairedDelimiter{\abs}{\lvert}{\rvert}
\DeclarePairedDelimiter{\norm}{\lVert}{\rVert}

\DeclarePairedDelimiter{\ceil}{\lceil}{\rceil}
\DeclarePairedDelimiter{\floor}{\lfloor}{\rfloor}
\DeclarePairedDelimiter{\round}{\lfloor}{\rceil}

\newsavebox\diffdbox
\newcommand{\slantedromand}{{\mathpalette\makesl{d}}}
\newcommand{\makesl}[2]{%
\begingroup
\sbox{\diffdbox}{$\mathsurround=0pt#1\mathrm{#2}$}%
\pdfsave
\pdfsetmatrix{1 0 0.2 1}%
\rlap{\usebox{\diffdbox}}%
\pdfrestore
\hskip\wd\diffdbox
\endgroup
}
\newcommand{\dd}[1][]{\ensuremath{\mathop{}\!\ifstrempty{#1}{%
\slantedromand\@ifnextchar^{\hspace{0.2ex}}{\hspace{0.1ex}}}%
{\slantedromand\hspace{0.2ex}^{#1}}}}
\ProvideDocumentCommand\dv{o m g}{%
  \ensuremath{%
    \IfValueTF{#3}{%
      \IfNoValueTF{#1}{%
        \frac{\dd #2}{\dd #3}%
      }{%
        \frac{\dd^{#1} #2}{\dd #3^{#1}}%
      }%
    }{%
      \IfNoValueTF{#1}{%
        \frac{\dd}{\dd #2}%
      }{%
        \frac{\dd^{#1}}{\dd #2^{#1}}%
      }%
    }%
  }%
}
\providecommand*{\pdv}[3][]{\frac{\partial^{#1}#2}{\partial#3^{#1}}}
%  - others
\DeclareMathOperator{\Lap}{\mathcal{L}}
\DeclareMathOperator{\Var}{Var} % varience
\DeclareMathOperator{\Cov}{Cov} % covarience
\DeclareMathOperator{\E}{E} % expected

% Since the amsthm package isn't loaded

% I prefer the slanted \leq
\let\oldleq\leq % save them in case they're every wanted
\let\oldgeq\geq
\renewcommand{\leq}{\leqslant}
\renewcommand{\geq}{\geqslant}

% % redefine matrix env to allow for alignment, use r as default
% \renewcommand*\env@matrix[1][r]{\hskip -\arraycolsep
%     \let\@ifnextchar\new@ifnextchar
%     \array{*\c@MaxMatrixCols #1}}


%\usepackage{framed}
%\usepackage{titletoc}
%\usepackage{etoolbox}
%\usepackage{lmodern}


%\patchcmd{\tableofcontents}{\contentsname}{\sffamily\contentsname}{}{}

%\renewenvironment{leftbar}
%{\def\FrameCommand{\hspace{6em}%
%		{\color{myyellow}\vrule width 2pt depth 6pt}\hspace{1em}}%
%	\MakeFramed{\parshape 1 0cm \dimexpr\textwidth-6em\relax\FrameRestore}\vskip2pt%
%}
%{\endMakeFramed}

%\titlecontents{chapter}
%[0em]{\vspace*{2\baselineskip}}
%{\parbox{4.5em}{%
%		\hfill\Huge\sffamily\bfseries\color{myred}\thecontentspage}%
%	\vspace*{-2.3\baselineskip}\leftbar\textsc{\small\chaptername~\thecontentslabel}\\\sffamily}
%{}{\endleftbar}
%\titlecontents{section}
%[8.4em]
%{\sffamily\contentslabel{3em}}{}{}
%{\hspace{0.5em}\nobreak\itshape\color{myred}\contentspage}
%\titlecontents{subsection}
%[8.4em]
%{\sffamily\contentslabel{3em}}{}{}  
%{\hspace{0.5em}\nobreak\itshape\color{myred}\contentspage}



%%%%%%%%%%%%%%%%%%%%%%%%%%%%%%%%%%%%%%%%%%%
% TABLE OF CONTENTS
%%%%%%%%%%%%%%%%%%%%%%%%%%%%%%%%%%%%%%%%%%%

\usepackage{tikz}
\definecolor{doc}{RGB}{0,60,110}
\usepackage{titletoc}
\contentsmargin{0cm}
\titlecontents{chapter}[3.7pc]
{\addvspace{30pt}%
	\begin{tikzpicture}[remember picture, overlay]%
		\draw[fill=doc!60,draw=doc!60] (-7,-.1) rectangle (-0.9,.5);%
		\pgftext[left,x=-3.5cm,y=0.2cm]{\color{white}\Large\sc\bfseries Chapter\ \thecontentslabel};%
	\end{tikzpicture}\color{doc!60}\large\sc\bfseries}%
{}
{}
{\;\titlerule\;\large\sc\bfseries Page \thecontentspage
	\begin{tikzpicture}[remember picture, overlay]
		\draw[fill=doc!60,draw=doc!60] (2pt,0) rectangle (4,0.1pt);
	\end{tikzpicture}}%
\titlecontents{section}[3.7pc]
{\addvspace{2pt}}
{\contentslabel[\thecontentslabel]{2pc}}
{}
{\hfill\small \thecontentspage}
[]
\titlecontents*{subsection}[3.7pc]
{\addvspace{-1pt}\small}
{}
{}
{\ --- \small\thecontentspage}
[ \textbullet\ ][]

\makeatletter
\renewcommand{\tableofcontents}{%
	\chapter*{%
	  \vspace*{-20\p@}%
	  \begin{tikzpicture}[remember picture, overlay]%
		  \pgftext[right,x=15cm,y=0.2cm]{\color{doc!60}\Huge\sc\bfseries \contentsname};%
		  \draw[fill=doc!60,draw=doc!60] (13,-.75) rectangle (20,1);%
		  \clip (13,-.75) rectangle (20,1);
		  \pgftext[right,x=15cm,y=0.2cm]{\color{white}\Huge\sc\bfseries \contentsname};%
	  \end{tikzpicture}}%
	\@starttoc{toc}}
\makeatother


\newcommand{\inv}{^{-1}}
\newcommand{\defi}{\equiv}
\newcommand{\gc}{^\circ}
\newcommand{\ii}{\item}
\newcommand{\ssi}{\leftrightarrow}
\newcommand{\sie}{\rightarrow}
\newcommand{\opname}{\operatorname}
\DeclareMathOperator{\img}{Im}
\DeclareMathOperator{\dom}{Dom} 
\DeclareMathOperator{\tr}{Tr}
\newcommand{\eps}{\epsilon}

\newcommand{\Ab}{\mathbb{A}}	\newcommand{\Bb}{\mathbb{B}}
\newcommand{\Cb}{\mathbb{C}}	\newcommand{\Db}{\mathbb{D}}
\newcommand{\Eb}{\mathbb{E}}	\newcommand{\Fb}{\mathbb{F}}
\newcommand{\Gb}{\mathbb{G}}	\newcommand{\Hb}{\mathbb{H}}
\newcommand{\Ib}{\mathbb{I}}	\newcommand{\Jb}{\mathbb{J}}
\newcommand{\Kb}{\mathbb{K}}	\newcommand{\Lb}{\mathbb{L}}
\newcommand{\Mb}{\mathbb{M}}	\newcommand{\Nb}{\mathbb{N}}
\newcommand{\Ob}{\mathbb{O}}	\newcommand{\Pb}{\mathbb{P}}
\newcommand{\Qb}{\mathbb{Q}}	\newcommand{\Rb}{\mathbb{R}}
\newcommand{\Sb}{\mathbb{S}}	\newcommand{\Tb}{\mathbb{T}}
\newcommand{\Ub}{\mathbb{U}}	\newcommand{\Vb}{\mathbb{V}}
\newcommand{\Wb}{\mathbb{W}}	\newcommand{\Xb}{\mathbb{X}}
\newcommand{\Yb}{\mathbb{Y}}	\newcommand{\Zb}{\mathbb{Z}}
%----------------------Blackboard font--------------------
\newcommand{\cA}{\mathscr{A}}	
\newcommand{\cB}{\mathscr{B}}
\newcommand{\cC}{\mathscr{C}}
\newcommand{\cD}{\mathscr{D}}
\newcommand{\cE}{\mathscr{E}}
\newcommand{\cF}{\mathscr{F}}
\newcommand{\cG}{\mathscr{G}}
\newcommand{\cH}{\mathscr{H}}
\newcommand{\cI}{\mathscr{I}}
\newcommand{\cJ}{\mathscr{J}}
\newcommand{\cK}{\mathscr{K}}
\newcommand{\cL}{\mathscr{L}}
\newcommand{\cM}{\mathscr{M}}
\newcommand{\cN}{\mathscr{N}}
\newcommand{\cO}{\mathscr{O}}
\newcommand{\cP}{\mathscr{P}} 
\newcommand{\cQ}{\mathscr{Q}}
\newcommand{\cR}{\mathscr{R}}
\newcommand{\cS}{\mathscr{S}}
\newcommand{\cT}{\mathscr{T}}
\newcommand{\cU}{\mathscr{U}}
\newcommand{\cV}{\mathscr{V}}
\newcommand{\cW}{\mathscr{W}}
\newcommand{\cX}{\mathscr{X}}
\newcommand{\cY}{\mathscr{Y}}
\newcommand{\cZ}{\mathscr{Z}}
%-----------------------Caligraphic font--------------------
%Captital Letters
\newcommand{\bmA}{\boldsymbol{A}}	\newcommand{\bmB}{\boldsymbol{B}}
\newcommand{\bmC}{\boldsymbol{C}}	\newcommand{\bmD}{\boldsymbol{D}}
\newcommand{\bmE}{\boldsymbol{E}}	\newcommand{\bmF}{\boldsymbol{F}}
\newcommand{\bmG}{\boldsymbol{G}}	\newcommand{\bmH}{\boldsymbol{H}}
\newcommand{\bmI}{\boldsymbol{I}}	\newcommand{\bmJ}{\boldsymbol{J}}
\newcommand{\bmK}{\boldsymbol{K}}	\newcommand{\bmL}{\boldsymbol{L}}
\newcommand{\bmM}{\boldsymbol{M}}	\newcommand{\bmN}{\boldsymbol{N}}
\newcommand{\bmO}{\boldsymbol{O}}	\newcommand{\bmP}{\boldsymbol{P}}
\newcommand{\bmQ}{\boldsymbol{Q}}	\newcommand{\bmR}{\boldsymbol{R}}
\newcommand{\bmS}{\boldsymbol{S}}	\newcommand{\bmT}{\boldsymbol{T}}
\newcommand{\bmU}{\boldsymbol{U}}	\newcommand{\bmV}{\boldsymbol{V}}
\newcommand{\bmW}{\boldsymbol{W}}	\newcommand{\bmX}{\boldsymbol{X}}
\newcommand{\bmY}{\boldsymbol{Y}}	\newcommand{\bmZ}{\boldsymbol{Z}}
%Small Letters
\newcommand{\bma}{\boldsymbol{a}}	\newcommand{\bmb}{\boldsymbol{b}}
\newcommand{\bmc}{\boldsymbol{c}}	\newcommand{\bmd}{\boldsymbol{d}}
\newcommand{\bme}{\boldsymbol{e}}	\newcommand{\bmf}{\boldsymbol{f}}
\newcommand{\bmg}{\boldsymbol{g}}	\newcommand{\bmh}{\boldsymbol{h}}
\newcommand{\bmi}{\boldsymbol{i}}	\newcommand{\bmj}{\boldsymbol{j}}
\newcommand{\bmk}{\boldsymbol{k}}	\newcommand{\bml}{\boldsymbol{l}}
\newcommand{\bmm}{\boldsymbol{m}}	\newcommand{\bmn}{\boldsymbol{n}}
\newcommand{\bmo}{\boldsymbol{o}}	\newcommand{\bmp}{\boldsymbol{p}}
\newcommand{\bmq}{\boldsymbol{q}}	\newcommand{\bmr}{\boldsymbol{r}}
\newcommand{\bms}{\boldsymbol{s}}	\newcommand{\bmt}{\boldsymbol{t}}
\newcommand{\bmu}{\boldsymbol{u}}	\newcommand{\bmv}{\boldsymbol{v}}
\newcommand{\bmw}{\boldsymbol{w}}	\newcommand{\bmx}{\boldsymbol{x}}
\newcommand{\bmy}{\boldsymbol{y}}	\newcommand{\bmz}{\boldsymbol{z}}

\title{\Huge{Mecánica Cuántica}\\Apuntes}
\author{\huge{Angel Almonacid}}
\date{\today}

\begin{document}
  \maketitle

\chapter{Aspectos Básicos: Función de Onda y Espacios Vectorial}
 \section{Función de Onda}
\subsection{Ecuación de Schrodinger}
Se parte de una hipótesis que cumpla $p(\vec{x},t)d^3x=\mid \psi(\vec{x},t)\mid^2d^3x$, siendo $\psi(\vec{x},t)$ la función que describe a la onda estudiada. De esta forma vamos a buscar que la función de onda cumpla cuatro condiciones:
\begin{itemize}
    \item $\frac{d^2x}{dt^2}=F/m$ Ecuación de movimiento \item $\frac{\partial}{\partial t} \psi(\vec{x},t)=L\psi+c$ Ecuación diferencial lineal \item $\int \left\lvert \psi(\vec{x},t) \right\rvert^2 d^3x=1 $ Función normalizada \item $\psi( \vec{x},t)=Ae^{i(\vec{k}\cdot \vec{x}-w\cdot t)}$ Forma sinusoidal
\end{itemize}

Así, podemos formular una hipótesis que además de las condiciones previas vuelva adimensional el exponente de la función

\begin{equation*}
    \psi(\vec{x},t)=Ae^{\frac{i}{\hbar}\left( \vec{p} \cdot \vec{x}- \frac{p^2}{2m}t\right)}
\end{equation*}

Ingresaremos la hipótesis en las ecuaciones planteadas para terminar de formular la función de onda

\begin{align*}
    \frac{\partial \psi}{\partial t}=\frac{-ip^2}{2\hbar m} \psi && \nabla\psi=\frac{i}{\hbar}\vec{p}\psi \rightarrow \nabla^2\psi=-\frac{p^2}{\hbar^2}\psi
\end{align*}

\begin{equation}\label{Schrodinger}
    i\hbar\frac{\partial}{\partial t} \psi=-\frac{\hbar^2}{2m}\nabla^2\psi
\end{equation}

Así, obtenemos la ecuación de Schrödinger para una función de onda plana. Si sabemos que una onda de esta forma es solución de la ecuación diferencial, una combinación lineal de estas ondas planas también lo será, así una solución general puede describirse de la forma (de manera discreta y continua respectivamente)
    $$\psi_k(\vec{x},t)=\sum_{1}^{n}e^{\frac{i}{\hbar}\left(\vec{p}\cdot\vec{x}-\frac{p^2}{2m}t\right)}\varphi(p)\frac{d^3 p}{(2 \pi \hbar)^3}$$
\begin{equation}\label{Superposition}
    \psi(\vec{x},t)=\int e^{\frac{i}{\hbar}\left(\vec{p}\cdot\vec{x}-\frac{p^2}{2m}t\right)}\varphi(p)\frac{d^3 p}{(2 \pi \hbar)^3}
\end{equation}

Donde $\varphi(p)$ es una función de peso para cada momento $p$, esta función de peso nos da información sobre la distribución del paquete de ondas. Para una distribución gaussiana, podemos hacer el cálculo de la constante de normalización $A$ y de la función de onda $\psi(\vec{x},t)$.Podemos definir la densidad de probabilidad $\rho(\vec{x},t)=\left\vert \psi(\vec{x},t) \right\vert^2 $ y el valor esperado $\langle x\rangle=\int_{-\infty}^{\infty}\left\vert \psi \right\vert^2 x dx$. Si deseamos hallar la desviación cuadrática media, sabemos que desde el estudio estadístico se define como $\left(\Delta x^2\right)=\sum_{i}^{n} \left(x-\bar{x}\right)^2$, así podemos definir la desviación cuadrática media como
 \begin{equation}
    \left( \Delta x^2\right)=\langle(x-\langle x\rangle)^2 \rangle
 \end{equation}
\section{Espacio de momentos}

Vamos a realizar la transformación al espacio de momentos a través de la transformada de Fourier, así podemos definir la siguiente transformación:

\begin{align*}
    \left\vert \psi (\vec{x},t)\right\vert^2 d^3x \longrightarrow \mathbf{W} (\vec{p},t)d^3p && \int_{V} \mathbf{W}(\vec{p},t)d^3p=1
\end{align*}

La relación de transformación a través de la transformada integral de Fourier se define de la siguiente forma
\begin{equation}\label{TFourier}
    \psi(\textbf{x},t)=\int\frac{d^3p}{(2\pi\hbar)^3}\varphi(\textbf{p},t)e^{\frac{i}{\hbar}(\textbf{p}\cdot\textbf{x})}
\end{equation}

Siendo $\varphi(\textbf{p},t)$ la transformada de Fourier de la función $\psi(\textbf{x},t)$. Para encontrar $\varphi$ debemos hallar la función inversa de Fourier, multiplicaremos por $e^{-\frac{i}{\hbar}p'x}$ e integraremos con respecto a x 
\begin{align*}
    \int d^3x \psi(\textbf{x},t)e^{-\frac{i}{\hbar}\textbf{p}'-\textbf{x}}=\int d^3p\varphi(\textbf{p},t) \frac{1}{(2\pi\hbar)^3}\int d^3x e^{\frac{i}{\hbar}(\textbf{p}-\textbf{p}')\cdot \textbf{x}}=\int d^3p \varphi(\textbf{p},t) \delta(\textbf{p}-\textbf{p}')=\varphi(\textbf{p}',t)
\end{align*}
Obtenemos como resultado la siguiente relación de transformación inversa, identificando la delta de Dirac dentro del termino de la integral con respecto a p
\begin{equation}\label{IFourier}
    \varphi(\textbf{p},t)=\int d^3x\psi(\textbf{x},t)e^{-\frac{i}{\hbar}(\textbf{p}\cdot\textbf{x})}
\end{equation}

\subsection{Densidad de probabilidad en el espacio de momentos}
Usaremos el mismo método para calcular la función de onda en el espacio de momento para calcular la densidad de probabilidad en este mismo espacio
\begin{align*}
\left\vert \psi(\textbf{x})\right\vert^2&=\int \frac{d^3 p}{(2 \pi \hbar)^3}\int \frac{d^3 p'}{(2 \pi \hbar)^3} \varphi(\textbf{p},t)\varphi^{\ast}(\textbf{p}',t) e^{\frac{i}{\hbar}(\textbf{p}-\textbf{p}')\cdot \textbf{x}}\\
\int \left\vert \psi(\textbf{x})\right\vert^2 d^3 x &= \int \frac{d^3 p}{(2 \pi \hbar)^3}\int \frac{d^3 p'}{(2 \pi \hbar)^3} \varphi(\textbf{p},t)\varphi^{\ast}(\textbf{p}',t)\int e^{\frac{i}{\hbar}(\textbf{p}-\textbf{p}')\cdot x} d^3 x\\
&=\int \frac{d^3 p}{(2 \pi\hbar)^3}\left\vert\varphi(\textbf{p}',t)\right\vert^2
\end{align*}

Así podemos identificar la densidad de probabilidad transformada al espacio de momentos como 

\begin{equation}
    \mathbf{W}(\textbf{p},t)=\frac{\left\vert\varphi(\textbf{p},t)\right\vert^2}{(2\pi\hbar)^3}
\end{equation}

Y definir valores esperados y desviaciones cuadráticas medias de la misma manera que en el espacio de coordenadas

\begin{align*}
    \langle \textbf{p} \rangle &= \int \frac{d^3p}{(2\pi\hbar)^3} \left\vert \varphi(\textbf{p},t) \right\vert^2 \textbf{p} \\
    (\Delta\textbf{p})^2 &=\langle(p-\langle p\rangle)^2\rangle=\int dp \mathbf{W}(p,t)(p-p_0)^2\\
\end{align*}

\subsubsection{Valor esperado del momento}
\begin{align*}
    \langle \textbf{p} \rangle &= \int \frac{d^3 p}{(2\pi\hbar)^3} \varphi(p,t)p\varphi(p,t)*= \int\int\int \frac{d^3p}{(2\pi\hbar)}d^3xd^3x'e^{\frac{i}{\hbar}p\cdot x'}\psi(x',t)*pe^{-\frac{i}{\hbar}p\cdot x}\psi(x,t)\\
    &=\int\int\int \frac{d^3p}{(2\pi\hbar)^3}d^3x d^3x'e^{\frac{i}{\hbar}p\cdot x'}\psi(x',t)*(-\frac{\hbar}{i}\nabla e^{-\frac{i}{\hbar}p\cdot x})\psi(x,t)\\
    &=\int\int d^3xd^3x'\psi(x',t)*(\frac{h}{i}\psi(x,t))\int \frac{d^3p}{(2\pi\hbar)^3}e^{\frac{i}{\hbar}p\cdot(x'-x)}\\
    \langle \textbf{p} \rangle &=\int d^3x \psi(x,t)*(\frac{h}{i}\nabla\psi(x,t))=\int d^3x\psi(x,t) p
\end{align*}

Así identificamos al operador momento en el espacio de las coordenadas
\begin{equation}
    p\rightarrow \frac{\hbar}{i}\nabla
\end{equation}

\section{Notación de Dirac y espacios de Hilbert}
\subsubsection{Bras, kets, operadores y producto escalar}
\textbf{Notación:} Vector en un espacio vectorial complejo de dimensión n:
\begin{multicols}{2}
    \begin{equation*}
        \vert \nu \rangle \rightarrow \begin{pmatrix}
            \nu_1\\ \nu_2\\ .\\.\\.\\\nu_N
        \end{pmatrix}
    \end{equation*}
    Las componentes $\nu_i$ son números complejos y la dimensión N del sistema depende de la naturaleza del sistema. $\vert \nu \rangle $ define el estado de un sistema.
\end{multicols}

La dimensión del espacio está relacionada con los estados que puede elegir el sistema. Si el estado se describe por una cantidad continua la dimensión del espacio es infinita.

\subsubsection{Álgebra.} Las peraciones suma y multiplicación por escalar son cerradas.
\begin{align*}
    \vert \nu \rangle+ \vert u \rangle = \vert w \rangle,  \ \alpha\vert\nu\rangle=\vert\nu\rangle\alpha=\vert w \rangle, \ 0\vert v \rangle = \vert 0 \rangle.
\end{align*}

    Los vectores $\vert v \rangle$ y $\alpha \vert v \rangle$ representan un mismo estado.


\subsubsection*{Observables y Operadores}
Un observable se puede representar por un operador que actúa en el espacio vectorial. Para un ket, el operador actúa por la izquierda y el resultado de su operación es otro ket.\\
\textbf{Operador A} $\mathbf{A} \vert v \rangle=\vert w \rangle$\\
Eigenket (autovector) y autovalores de un operador 

\begin{align*}
    A \vert \lambda \rangle = \lambda \vert \lambda \rangle, \ A\vert\lambda_i\rangle= \lambda_i\vert\lambda_i\rangle. \ \lambda_i \in \mathbb{C}.
\end{align*}

\begin{gather*}
  \{ \lambda_i\} \rightarrow \text{Conjunto de autovalores del operador A.} 
\end{gather*}

Cualquier ket del espacio de estados de un sistema puede expresarse como una combinación de autovectores independientes (asociados a sus respectivos autovalores ${\lambda_i}$) de la forma

$$\vert \alpha \rangle = \sum_{i=1}^{N} c_i \vert \lambda_i \rangle, \text{ con } A \vert\lambda_i\rangle=\lambda_i\vert\lambda_i\rangle \text{y}  c_i \in \mathbb{C}$$





\section{Dinámica de Ecuación de Schrodinger}

% \begin{advertencia}
%     Suponemos conocidas las propiedades de la función 
%     \[
%         \func{\cos}{\R}{\R}
%         \texty
%         \func{\arccos}{[-1,1]}{\R}.
%     \]
% \end{advertencia}

\begin{align*}
    &\mid \alpha \rangle = \mathcal{U}(t,t_0) \mid \alpha,t_0 \rangle\\
    &i \bar{h}\frac{\partial}{\partial t} \mathcal{U}(t,t_0) =H\mathcal{U}(t,t_0) \rightarrow i \bar{h} \frac{\partial}{\partial t}\psi
\end{align*}

\chapter{Operadores compatibles y Relaciones de incerteza}
Para la descripción de sistemas no unidimensionales, las cantidades determindas por el operador $\hat{H}$ no son suficientes para describir el sistema, ante esta necesidad 
\section{Relación de Incerteza}
Definiremos el operador $\sigma(A)\equiv A- \langle A \rangle$. Y la dispersión $\sigma(A)^2$, que cumple:
\[
  \langle\sigma(A)^2\rangle= \langle(\hat{A}- \langle A\rangle)^2\rangle=\langle \hat{A}^2\rangle -\langle \hat{A} \rangle ^2
\]
Aplicando la desigualdad de Schwartz a dos operadores hermíticos que forman estados: $\sigma(A)\ket{\psi}, \sigma(B)\ket{\psi}$. Adicionalmente el producto de cualquier par de operadores puedes escribirse: $\sigma(A)\sigma(B)=\frac{1}{2}\left(\left[ \sigma{A},\sigma{B}\right]+\{\sigma(A),\sigma{B}\} \right)$ para separarloen sus partes real e imaginaria, así mismo los corchetes y llaves cumplen:  $\left[ \sigma{A},\sigma{B}\right]^\dagger=-\left[ \sigma{A},\sigma{B}\right]$ y $\{\sigma(A),\sigma{B}\}^\dagger=\{\sigma(A),\sigma{B}\}$. Siendo $\{  \}= \frac{1}{i \hbar } \left[ ,  \right]$.

\begin{align*}
  \bra{\psi}\sigma(A)^2\ket{\psi}\bra{\psi}\sigma(&B)^2\ket{\psi}\geq \left|\bra{\psi}\sigma(A)\sigma(B)\ket{\psi}\right|^2\\
  \left| \bra{\psi}\sigma(A)\sigma(B)\ket{\psi} \right|^2&=\frac{1}{4}\bra{\psi}\left\{ \sigma(A),\sigma(B)  \right\}+\frac{1}{4}\left| \bra{\psi}\left[ \sigma(A),\sigma(B)  \right] \ket{\psi} \right|^2\\
\end{align*}
Aplicando esto al operador desviación, tenemos:
\begin{equation}
\sigma(A)^2 \sigma(B)^2 \geq \frac{1}{2}\left| \langle \left[ A,B  \right] \rangle  \right| 
\end{equation}
\end{document}
