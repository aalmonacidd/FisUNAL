\documentclass[a4,11pt]{aleph-notas}
% Se recomienda leer la documentación de esta
% clase en https://www.alephsub0.org/recursos/

% -- Paquetes adicionales
\usepackage{aleph-comandos}
\usepackage{enumitem}
\usepackage{amssymb}


% -- Datos de las notas
\universidad{Universidad Nacional de Colombia}
\autor{Angel Almonacid}
\materia{Termodinámica}
\nota{}
\fecha{\today}

\longtitulo{0.65\linewidth}
\logouno[3.0cm]{Logos/LogoUnal}
\logodos[2.0cm]{Logos/IconoFisica}


% -- Comandos adicionales


\begin{document}

\encabezado

\vspace*{-8mm}
\section{Ciclos}

\begin{advertencia}
    Sabemos que para cada ciclo termodinámico se tiene que:

    $$\Delta \mathbb{U}=0$$
\end{advertencia}

Recuerde el siguiente resultado.

\begin{teo}
    Sean $x\in \R$ y $y\in [-1,1]$. Se tiene que si
    \[
        y =\cos(x),
    \]
    entonces
    \[
        x = \arccos(y) + 2k\pi
        \texto
        x = -\arccos(y) + 2k\pi
    \]
    con $k\in\Z$.
\end{teo}

%%%%%%%%%%%%%%%%%%%%%%%%%%%%%%%%%%%%%%%%%%%%%%%%%%%%
%% Ejercicio 1
%%%%%%%%%%%%%%%%%%%%%%%%%%%%%%%%%%%%%%%%%%%%%%%%%%%%
\begin{ejer}
    Resolver la ecuación
    \[
        1=4\cos\left(\frac{x}{3}\right).
    \]
\end{ejer}

\begin{proof}[Solución]
    Tomemos la ecuación y dividamos entre 4, obtenemos la expresión equivalente:
    \[
        \frac{1}{4}=\cos\left(\frac{x}{3}\right),
    \]
    con lo cual, las soluciones son
    \begin{enumerate}[label=\textit{\alph*)}]
        \item\label{ej01:c01} $\displaystyle \frac{x}{3}=\arccos\left(\frac{1}{4}\right)+2k\pi$, con $k\in\Z$; o
        \item\label{ej01:c02} $\displaystyle \frac{x}{3}=-\arccos\left(\frac{1}{4}\right)+2k\pi$, con $k\in\Z$.
    \end{enumerate}
    Así, la solución de la ecuación es
    \[
        x=3\arccos\left(\frac{1}{4}\right)+6k\pi
        \qquad\text{o}\qquad
        x=-3\arccos\left(\frac{1}{4}\right)+6k\pi.\qedhere
    \]
\end{proof}


\end{document}
